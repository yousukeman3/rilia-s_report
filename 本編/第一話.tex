\documentclass[oneside, a4paper]{jsbook}
\usepackage[textwidth=50zw,lines=43]{geometry}
\usepackage{bxpapersize}
\usepackage{pxrubrica}
\usepackage{okumacro}
\usepackage{titlesec}
\usepackage{epigraph}
\pagestyle{headings}
\renewcommand{\postchaptername}{\if@english\else 話\fi}

\begin{document}


\chapter{はじまり}
\epigraph{あの満天に広がる星空は、きっと女神様の魔法だったんです――}{リリアーナ・クレール}

\section{プロローグ}

その人は\kenten{アレ}の前に立ち塞がりました。\\

当時の私はアレが何なのか全然知らなかったけど、きっととても強大で邪悪なものだということは分かりました。\\
あの時、アレは――悪竜の群れは明らかに悪意を以って、森を焼き、家屋を踏み潰し、人を食らい、命を蹂躙していたのです。\\
悪竜は女の人も、子供も、お爺ちゃんお婆ちゃんも見境なく殺していきました。
村の腕利きの人たちも、悪竜から皆を守って死にました。
仲のいい友達も何人も居なくなって、お父さんも、私とお母さんを庇って…… 。\\

%\newpage

その時です、\kenten{その人}は私たちの前に現れ、アレとの間に立ちふさがりました。\\
その人が張った結界によって、森は不思議な守りに包まれます。\\
あらゆる一切の邪悪、不条理、暴虐を断つ、清浄なる夢幻の守り。\\
それは悪竜であっても例外なく防ぎました。

悪竜は困惑します。\\
『この結界は何なのか、何がここまで強力な結界を作り上げたのか。』

そんな中、その人は臆することなく悪竜の群れへと突き進みます。\\
その人の杖の一振りは、青白い弧を描き、その軌跡は一条の星となって次々と悪竜を撃ち落としていきます。

一匹、また一匹と打ち落とされていく中、悪竜は理解します。\\
『自分たちを打ち落としているのは、あの村を守っているのは、あの人間なのだ。』と。\\

%\newpage

悪竜の群れはその人を倒さんと向かっていきます。\\
ある悪竜は鋭い牙で食らいつきました。\\
ある悪竜は鉄より硬い爪を突き立てました。\\
ある悪竜は灼熱の業火で焼き払いました。\\
――数百の悪竜がその人へと攻めかかりました。\\

それでも――それでも、その人は怯みません。\\
道を阻む悪竜を次々と打ち砕き――\\
一歩、また一歩と群れの中心へと突き進んでいきました。\\

そして――その悪竜の群れの主と相対するのです。\\

――その時は来ました。\\
それは、星空そのものでした。\\

\noindent
「其は虚無の海に輝く星。途絶えず、揺るがず、違えず、人を照らす希望の光」\\

その時の光景を忘れたことはありません。\\
真昼の空に広がるのは満天の星空。\\
今まで見たどんな景色より、それは美しく鮮烈で、そして何より――\\\\
「彼の者を導き、その道行に眩く奇跡と祝福を――!!」\\\\
私の心を掴んで離しませんでした。\\
だから私は――
\newpage
\section{巣立ち}

\noindent
「お財布は持った?」\\
「持ったよ!」\\
「お弁当は持った?」\\
「持った!」\\
「\ruby{身分手帳}{クラスブック}は?」\\
「持ったってば!!」\\
「じゃあ――ティリア先生の推薦書は!?」\\
「そんなの絶対忘れるわけないよ!!」\\

その日は私が村を出る日でした。
家の前には、旅立つ私を見送るために村のみんなが集まっています。\\
私はこの春、ローゼ=フィアス魔術学園に入学します。
あの格式高いローゼ=フィアス魔術学園です(と言ってもこの時点ではどれぐらい歴史があってすごいところかとかは知らなかったんですが……)。\\
はい、つまり私は――\\

――これから魔術師を目指します。\\

家事もだめ、勉強もだめ、狩りも農業も商売もダメな私が初めて頑張れそうな事が出来たって言ったら、村のみんなはすごく喜んでくれました。\\
村はまだあの事件の傷が癒えていなくて、生活も苦しいはずなのに、学校への入学料は村のみんなで出すって言って、私の遠慮も聞き入れてもらえませんでした。\\
この日見送りに来てくれてた人たちはみんな、こんな身勝手な私の幸せを願ってくれる、私の大好きな人たちです。私はそんなみんなに報いたくて、立派な魔術師になって、村のみんなを助けてあげられるように頑張りたいと考えていました。――なんていうのは本当は詭弁なのですが。\\

そんな中、まだかまだかとみんなを待たせて、私とお母さんは何をしていたかというと、最後の持ち物チェックをしていました。\\
ドジで間抜けな私は、日頃からすぐに忘れ物をするので、こうしてお母さんと一緒に持ち物を確認するのは、外出する時の習慣だったのです。\\
でも、それもこれまで。
これからは自分で用意をしなければいけません。(といっても、結局寮ではエリーと一緒に確認することになるんだけど。)\\

「制服よし、髪型よし、笑顔よし!」\\
私とお母さんで、いつもの確認を一緒に言い終えたら、最後の準備はおしまい。\\
玄関で靴を履いて、いつもの靴ベラとお別れを済ませたら、ドアを開けてみんなの元へ。\\

\noindent
「って、おわっと」\\

……このようにつまずいてしまうこともあります。\\

\noindent
「いつも歩いてる玄関でよくも器用につまずけるわねぇ」\\
「いったいよぉ……なんで今日に限ってつまずくのかなぁ」\\
「別に今日だけの話じゃないでしょう。さっさ、皆いるわよ。挨拶しなさい」\\

ちなみにこのように私はいろんな所でつまずくいては倒けています。(今日に限ってと言うのもただの見栄です。ごめんなさい。)\\

\noindent
「いたそうだなぁリリア。こんな調子じゃァ学園でやってけねぇだろ」\\
「茶化さないであげてよおじさん。リリアーナちゃん、大丈夫?怪我したところあったらすぐに見せてね?」\\

このふたりは鍛冶師のバレルさんと薬師で治療術師のローレルちゃん。ローレルちゃんはよく怪我をする私をお薬や魔術で治してくれます。すごく立派な子で、私とそんなに歳も離れていないのに、既に治癒師の仕事をしっかりとこなしています。道中で怪我をしても大丈夫なように、昨日たんまりとお薬を渡してくれました。

バレルさんもこんな感じではありますがいつも面倒を見てくれている良い人で、学園に入学すると聞くなりスタッフを作ってくれました。スタッフは素材の都合上すごくお金がかかるのに、タダで戴いてしまいました。\\

\noindent
「大丈夫だよローレルちゃん、ちょっと擦っただけだしほとんど怪我になってないから」\\
「おいおい、おじさんにはスルーかよ。ちょっと悲しくなっちまうぜ」\\
「意地悪なことを言うおじさんは知りません」\\
「おじさんも心配しているだけだよ。そう気を悪くしないであげて?」\\
「まぁ、ローレルちゃんがそこまで言うなら……。おじさんもスタッフありがとう、大事にします」\\
「いやぁ、別にそういう風に言わせたかった訳じゃなくってだなァ……いや、気に入ってくれたなら鍛冶屋冥利に尽きるってもんよ。頑張れよリリア!」\\

そういうとおじさんは私の背中を叩きます。(これごまかしたよね?絶対照れ隠し入ってるよね?)\\

\noindent
「これこれ、他にも話したいものは居るんじゃ、貴様らだけでリリアーナを独占するな」\\
「あ、長老様、ごめんなさい……」\\
「わりぃわりぃ、朝っぱらから心配になるような事しでかすんでついな」\\

この人は長老、村をまとめてくれている方です。(実はみんな長老としか呼ばないから未だに名前を知らないんだよね……。今度調べておきます……。)私が入学したいって言ったら、最初は反対していましたが、最後には納得してみんなに声をかけて入学金を募ったりしてくれました。普段もすごく優しい人で、困ったことがあったら相談に乗ってくれたりしてくれます。\\

\noindent
「全く、朝話し始めると出発が遅くなるから、言いたいことがあるなら昨日のうちに言っておけとあれほど言っておったのにな。ま、いい。リリアーナ、村を代表して儂が祝辞を言わせてもらうぞ」\\
「じいさんは良いのかよ」\\
「儂以外に誰か適任が居るんか?」\\
「いやバルディウスの兄貴とか」\\
「あやつ今居らんやろ。今日も見送りたいと言いながらも仕事で王都じゃろうが。というかそもそもリリアの伯父じゃろあやつ。絶対伯父バカを炸裂させて終わりじゃ」\\
「まとめ役っつったら浮かんだ名前を言っただけなんだけどな……」\\
「うるさいうるさい。お前と話しとったら埒が明かんわ。」\\

此処までの話で分かると思いますが、私の村の人はみんな仲良しです。誰かのピンチはみんなで乗り越えて、誰かの嬉しいことはみんなでお祝いします。私はこの時そんなみんなに沢山助けてもらって、このチャンスにありつけました。(ちなみに伯父さんの話は後でするからもうちょっと待ってね)\\

\noindent
「こほん、では改めて。リリアーナ・クレール、この度は村の――」\\
「ごめんね村長、今日は長いやつはいいなかぁ~?なんて」\\
「コラ、お前はまたそうやって……はぁ、ま確かに、今更お前にこういう事を言ってもお前からすれば門出が遅くなるだけなんじゃろ……」\\
「そういう事だよ!流石村長、話が分かるね!」\\
「バカモン!皮肉なのが分らんのか……はぁ。では短く纏めるぞ、村のモンはみんなお前が大きくなって帰ってくるのを応援しとる。やるべきことはしっかりとこなしながら、のびのび勉強するんじゃぞ」\\
「うん!みんなありがとう!!私行ってくる!!」\\

この日は行商人のダリアおじさんが街の方に帰る日なので、その馬車の荷台に乗せてもらいます。\\

\noindent
「ほら、皆とお別を済ませたらさっさと乗りなお嬢ちゃん」\\
「うん!!それじゃあ行ってくるね、みんな」\\
「いってらっしゃい、リリア。馬車に乗るときに体勢崩して倒けたりしなさんなよ」\\
「さすがにだいじょ――おっとっと」\\
「ほら言わんこっちゃない」\\

こうして、私は村を出ます。この一歩が、私の人生をどうやって変えるのかは、まだこの時にはよくわからないけど、新しい場所、新しい人とたくさんの発見が待ち構えていることは言うまでもなく、この時の私は期待に胸を膨らませるのでした。

\newpage

\section{出会い}

カタカタと馬車に揺られて10分ほどが経ちました。森に囲まれた場所なので当然なのですが、村の家屋などは既にどこにも見当たりません。\\

私が生まれ育って、今から旅立つこの村はココン村。人口は200~300人程度の小さな村です。自然と調和するのどかな村で、名産品はなめし革、市場では上等品として取引されています。狩りをするため、強い狩人の方はたくさんいらっしゃるのですが、魔術師の方はほとんどいません。戦闘ができる魔術師に限れば一人も居ないのです。\\

先の惨劇では、村は大きな傷を負いました。あれから一年、みんなの尽力のお陰である程度村は整備されましたが、当時は家屋にも被害があり、外れに出れば焼け焦げたりなぎ倒された木々は未だにたくさん見当たります。

ここまでのことになったのは、村に戦える魔術師が居ない事も大きな要因でした。悪竜には刃が効きません。矢も鈍器も効きません。そうなのです、概念生物である悪竜にはたった一つの方法を除いて、攻撃が届かないのです。そして、その方法こそが魔術というわけです。\\

その様な話を知らないうちの狩人たちは、勇敢に、果敢に悪竜達に挑みましたが、結果としては返り討ち、いえ、おなじ土俵にすら立てないままに蹂躙されてしまいました。\\

では、わたしたちの村がなぜ辛うじて生き残れたかといえば、一重に流浪の魔術師様が村をお救いになられたお陰です。

そんな魔術ーーティリア先生と出会ったのは、ちょうどいま馬車が走っている林道でした。あまりにも衝撃的過ぎて、一言一句先生の言動を覚えています。\\\\\\

その時の私はローレルちゃんと一緒に薬草を集めるついでに散歩をしていました。\\

\noindent
「今日もありがとうリリアちゃん。お陰で今日もカゴいっぱいに薬草が取れたよ!」\\
「でも、薬草じゃない草もいっぱい取っちゃった気もするけどなぁ……絶対邪魔しちゃったよね……」\\

当時の私には野草の知識なんて無かったので、薬草の見分けなんかつくはずもありません。私はなんとなくや勘で見たことがある草を選ぶのが精一杯でした。\\

\noindent
「そんなことないよ、リリアちゃんのお陰で、私は見分けるだけで済んだし、何より一緒に出来て楽しかったもん」\\
「ローレルちゃんは良く出来た優しい子だよ……私なんて出来ること全然ないし、14にもなったっていうのに、未だにふらふら遊んでるだけだもん」\\
「そんなに焦らなくても大丈夫だよ。きっと、リリアちゃんにも出来ること絶対見つかるって。それよりも、今日のお散歩はどこまでいくの?あんまり遠くに行くとお母さんに怒られちゃうよ?」\\
「そうだなぁ……じゃあ桃の木のところまで行こう。じゃぁ、そうと決まれば競争!」\\
「待って、だめだよリリアちゃん!急に走ると倒けーー」\\
「うわっと」\\
「……大丈夫?リリアちゃん」\\

盛大に倒けました。今回は両手をつけたので、顔は無事でした。(因みに何もないところじゃなくて、ちゃんと小石につまずいて倒けただけだから、私の名誉は保たれるはずだよね?)

そして、ふと顔を横に向けたときでした。木々の隙間から漏れる天使の梯子の先に――\\

――昏い青色の女神様が、慈愛の笑みを浮かべて微睡んでいました。\\

その姿に見惚れてしまった私は、凍った時間に吸い込まれてしまうように、ただただ女性を見つめるしかありませんでした。\\

\noindent
「――ちゃん。リリアちゃん!!」\\
「え、あ……人が……倒れて……」\\
「人?……本当だ!様子、見に行ってくるね!」\\

かろうじて正気に戻った私は、ローレルちゃんの後を追い、女性を改めて覗き込みます\\

真っ直ぐに伸びた涼しげな青色の髪は腰まで届き、桃色に染まった頬とともに、雪の様に白い肌を際立たせています。また、近づいて初めて分かった事ですがとても背が高く、枝のように細い四肢とは対称的に豊満な胴がゆったりとした青色のドレスで包まれていました。どこか幼さを残しつつも大人びた顔立ちからは、穏やかな笑みを浮かべているにも関わらず氷の様な冷たさも感じられます。\\

私はずっと女性に心を奪われていましたが、ローレルちゃんが触診をしている姿をみて現実に引き戻されました。\\

\noindent
「はっ、ローレルちゃん、脈とかある!?ちゃんと生きてる!?」\\
「大丈夫、気絶してるだけみたい。見た限り傷とかもないよ」\\
「それじゃあ病気で倒れちゃったとか?」\\
「それか疲労とかかも。流浪の冒険者の方とかなら道中で疲れて倒れちゃう事とかあるって聞いたことあるよ!!」\\
「見るからに冒険者って格好じゃないけど……」\\
「そしたら尚の事だよ。馬車とかで来たならこんな所で倒れているわけ無いし……」\\

その時です。\\

ーーグウゥゥゥ

\noindent
「ん?」\\
「え、今のリリアちゃん?」\\
「違うよ!!あ、わかった!実はローレルちゃんなんでしょ!誤魔化そうとしてもだめだよ」\\
「わ、私じゃーー」\\

ーーグウゥゥゥゥゥゥゥ\\

\noindent
「……」\\
「……」\\

二人とも薄々分かっていました。この音は二人のお腹から発せられたものではありません。

でも、だからこそわかりませんでした。この音は一体どこから鳴っているのか。

周りには他に人はおろか、動物もいません。私達でも無いのであれば、一体どこから?

『全てのあり得ないことを除外して最後に残ったものが如何に奇妙なことであってもそれが真実となる』のなら、私達は次のことを認めざるを得ません。\\

\noindent
「もしかして、この人から?」\\
「えぇ……?」\\
「で、でもお腹が鳴ってるってことは元気な証拠だよ!もう一度意識を確認してみるね」\\

そう言って、ローレルちゃんが意識を確認するために顔に近づいたその時でした。彼女の目がカッと見開き、次の瞬間ーー\\

ーーかごの中に頭を突っ込みました。\\

\noindent
「「は?」」\\

頭が突っ込まれたかごからはムシャムシャという音が聞こえます。そして暫くの間、その音だけがこの場を支配していました。\\\\\\

あれから10分ほどが経ちました。\\

\noindent
「まずはお礼からですね!お腹が空いて動けなかったところを助けて頂いて、ありがとうございます!!」\\

かごの中の薬草をヤギのように全て食べつくすと、彼女は起き上がり私達に向き直って、そう言いました。改めて見てもその姿は見目麗しく、先程の許し難き蛮行と結びつける事はその上品な立ち居振る舞いからは非常に難しいところがありました。ですが、それはこの目で見たものを否定する根拠にたり得ません。\\

\noindent
「……ぃえ……お力になれたのなら良かったです……」\\
「あの……それ……ローレルちゃんがお薬作るために集めた薬草なんですけど……」\\
「あっ……」\\

彼女の顔が青ざめて行きます。このとき自分が失態を犯したことをやっと理解したのでしょう。そのうち、平謝りの姿勢に切り替わりました。\\

\noindent
「あっ……あの……その……えっと……ごめ、ごめんな……さいぃぃぃいいぃぃ」\\
「あ、えっと大丈夫ですよ。また集めればいいだけの話なので……」\\
「ごめんなさいごめんなさい。弁償させてください。食べた分全部私が用意します!!あ、それで足りなければなんでもさせてもらいます!!だからぁ……だからぁ……」\\

遂に『ぐすん、ひっく』と泣き始めてしまいました。\\

「えっと、と、とりあえず落ち着こうよ!ほんとに薬草は集めればいいだけだから!」\\
「そうですよ。それに、おなかが空いていらしたのでしたらお役に立ててよかったです……って、まって、あの量を全部食べてしまったいうことは……」\\
「うっ……うっぷ……おえぇぇぇ」\\

副作用です。薬効成分があるものを大量に摂取するとこうなります。結局彼女は調子を崩してしまったので、ローレルちゃんが様子を見てくれている間に私が村まで大人を呼びに行き、空き部屋がある私の家で休ませることになりました。最初に抱いた憧れは遠い彼方へと消え去り、残ったのは面白いお姉さんというレッテルだけです。\\

はい、これが私の人生にとても大きな影響を与えた恩師との出会いでした。\\\\\\

彼女が目を覚ましたのはあれから6時間ほど経った夕ご飯前のことです。ローレルちゃんが彼女に事の経緯を伝えると、家に受け入れてくれたお礼をしたいということで、私とお母さんを呼び出しました。\\

「こんばんは、お二方。まずはお礼とお詫びをさせていただきます。この度は私を家に引き受けてくださり、ありがとうございました。そして、とても恥ずかしい理由でご迷惑をお掛けすることになってしまい大変申し訳ありません。」\\
「いやいいのよ別に。空いてる部屋適当に貸してあげただけだし。それより大丈夫?倒れた人が居るって言うんで、急いでできる限りの掃除はしたけど……。」\\
「いえ、とてもぐっすり快適でした。」\\

顔色もすっかり良くなったのを見て、私達は安堵の息を吐きます。ローレルちゃん曰く、どうやらあの薬草はそのままではそこまで強い効能はない様です。容態的にもそこまで問題は無いことは分かっていたのですが、実際に元気になるまではやはり不安は無くならないものです。\\

\noindent
「恩人の皆さんに名乗りもしないのは失礼ですね。わたしはティリアシール・フリーマン。ティリアと呼んでくださると嬉しいです」\\
「ティリアさんと言うのね。私はユリアーネ・クレール。この子、リリアの母親よ」\\

お母さんは私の頭をくしゃくしゃと撫でながらそう言いました。それにならって私とローレルちゃんも挨拶をします。\\

\noindent
「私はリリアーナ・クレール。お母さんも言ったと思うけど、ユリアお母さんの娘です。リリアでいいよ!よろしくね」\\
「私の名前はローレルです。下の名前はありません。魔術師としては1代目で、血統としては浅いどころか全く積み上げのない者ですが、僭越ながら治療の方を施させていただきました」\\

砕けた挨拶をした私とは違い、ローレルちゃんはとても畏まった態度で挨拶をしていました。元々ローレルちゃんは言葉遣いや立ち居振る舞いも丁寧なところはありますが、いつもはここまで堅苦しくはありません。\\

\noindent
「ローレルちゃん?どうしてそんなガチガチなの?」\\
「えっと、それはね……」\\

ローレルちゃんが渋々といった感じで説明を始めようとしたその時、その言葉を遮る様にティリアさんが口を開きます。\\

「それは、わたしが偉大なるすごーい魔術師だからです。えっへん」\\

ローレルちゃんはその言葉に跪きましたが、それも相まって、私とお母さんは一瞬ポカンとした後、可笑しくなって笑ってしまいました。\\

\noindent
「あははは。い、偉大な魔術師様。あは、あははははは」\\
「偉大なすごーい魔術師様は薬草をむしゃむしゃヤギさんみたいに食べたりしないよ!あははははは」\\
「むぅぅぅぅぅ!失礼ですね!本当なんです!!信じて下さい!!」\\
「えっと、リリアちゃんもユリアさんも、ちゃんと聞いて!この人はーームッ!?」\\

ローレルちゃんは何かを言い掛けましたが、今度は急に口をもぞもぞさせながら黙り込んでしまいました。私はすかさずティリアさんに返します。\\

\noindent
「ほら、ローレルちゃんも黙り込んじゃったよ!」\\
「こら、リリア。私が言えたことじゃないけど、あんまり人を馬鹿にするのも良くないわよ」\\
「ごめんなさーい。そうだ、ローレルちゃん、ティリアさん、夕ご飯もう出来たからお腹が大丈夫なら一緒に食べよ?」\\
「むぅ……まぁ、せっかくのご厚意を無駄にはできませんので……」\\

お怒りのティリアさんでしたが、お母さんのご飯を食べるとすっかり機嫌を直してくれました。結局、この日ティリアさんはうちで泊まる事になり、次の日の朝ごはんも一緒することになりました。\\\\\\

\noindent
「おはようございます皆さん、今日もよろしくお願いしますね」\\

そういってティリアさんは起きて来ました。私とお母さんからの返しを聞いたのち、礼儀正しい所作でテーブルに着くと、私たちにこう続けました。\\

\noindent
「わたし、決めたのですが、ちょっとの間この村に滞在しようと思います」\\
「おや、ここに長居してもいいの?旅の方みたいだけど、もともと何か目的があるんじゃないのかしら?」\\
「うーんと、私は旅そのものが目的なので、特にやらなきゃいけないこととかは無いんです。それで今は、優しくしていただいた皆さんに恩返しがしたいなと」\\
「そう。そしたらどんなことを手伝ってもらおうかしら……そういえば、一昨日から近くの井戸の水が出なくなってしまって……力仕事で悪いんだけど、近くの川から水を汲んでくるのを手伝ってもらえないかしら」\\
「井戸の水が出ないんですか?」\\
「そうなのよ。あまり考えたくないんだけど、枯れちゃったのかもしれないわね……」\\

ティリアさんはそう聞くと、何かを思いついた様に言いました。\\

\noindent
「ちょっと、後でその井戸を見せてもらえませんか?」\\\\\\

さて、ティリアさんは井戸の前に行くなり、その辺で適当な木の棒を拾うと、井戸の周りの何か所かに魔方陣を書き始めました。魔方陣は、茶色や水色に光るのですが、場所によって輝く強さが違います。\\
私を含めた村の人達にとってはこういう光景は珍しく、みんなどんどん見物に集まってきます。
ティリアさんは周りの観衆にも気を向けずに、ブツブツと喋り始めました。\\
  
 「……水脈の流れ自体には異常ありませんね。どっちかというと\ruby{水元素}{ヒュードル}と\ruby{土元素}{ゲー}のバランスが悪いみたいです。具体的に言えば\ruby{土元素}{ゲー}が\ruby{水元素}{ヒュードル}を邪魔してしまっているので、性質を\ruby{乾}{かん}から\ruby{湿}{しつ}にいじってしまえばいいですね」\\

そう言うとティリアさんは持っていた木の棒を地面に平行に持つと、持っていた手を放してしまいました。

その瞬間、みんなはざわめきました。\\
本来なら重力に従って落ちる筈のそれは、なんと驚くべきことにーー\\

ーーそのまま宙に浮かんでいたのです。\\

彼女はそのざわめきすら気にならないかのようにその木の棒に腰掛けると、木の棒が独りでに動き出し、ティリアさんを井戸の中へと運んで行ってしまいました。

みんなが驚くのも無理はありません。私達の村には全く魔術師が居ないわけではありませんが、物が浮いている光景なんて誰も見たことがなかったのです。

残された私達は、ただただ井戸の中を覗くことしかできません。私達は結局ティリアさんが帰って来るまでの2分弱もの間その場に立ち尽くしていました。\\

\noindent
「皆さん、1回水を汲んでみてくれますか?」\\

\noindent
帰って来たティリアさんは、自信満々にそう言います。言われた通り試しに水を汲んで見れば、きちんと水が出てくるわけです。もちろん、村の人は大喜びで、中にはティリアさんを担ぎ上げてしまう人たちもいました。\\

ティリアさんもまんざらでは無い様でーー\\

\noindent
「ふふふ……苦しゅうない、苦しゅうないですよ。何ならもっと褒めてくれてもいいです!どうですか皆さん、これが偉大なるすごーい魔術師の力なんです!!ふふーん!」\\

ーーと楽しそうにしていました。\\

斯くして、村の人気者へなった彼女には様々な依頼が舞い込むことになります。失せ物探し、魔物狩り、天気などの占い、素材の調達等、彼女はこれらを全て無償でしかも楽しそうにこなしていくわけです。

彼女は、寝泊まりをしているここを拠点に活動をするので、私の家はすっかり魔術屋さんとなってしまいました。私はいつも通り過ごしながらもティリアさんがこなすいろんなことに興味を持つようになりました。\\\\\\

そんな日々が2週間ほど続いたある日、なんでも無い賑やかな昼下がりのことでした。

前日からティリアさんはローレルちゃんを初めとした用事がある人達を連れて、隣の町まで希少な薬品の素材を買い出しに行っていました。隣町といっても、うちの村は辺鄙な所にあるので、片道だけで一日ぐらいかかります。\\

私はといえば、ティリアさんもローレルちゃんも居ないので、村長に構って貰っていました。\\

\noindent
「暇だね〜〜」\\
「暇なんてあるものか。そう感じるのはお前が山のようにある困りごとに目を向けんからじゃろがい。やりたいこともないなら、いい加減自分のできることを見つけて、少しは村の役に立とうとせんかい」\\

私に返事をしながら、村長はずっといろんな書類とにらめっこしています。\\

\noindent
「じゃあなんか紹介してよ〜〜。私何やったって上手く行かないんだもーん」\\
「お前はやる気が足りんのじゃ。何かで躓いたらす〜ぐ諦めて。だいたいお前はなぁ……ん?」\\
「どうしたの?」\\
「外が騒がしぃなっとらんか?」\\
「え?」\\

村長に言われて私は外の様子を見に行くことにしました。すると、みんなは頻りに南東の空を見ながら、『あれは何なのか』等と、ざわめいていました。

実際に眺めてみると、空には数え切れないほどのいくつもの黒い斑点が見えます。みんなはあれについて会話を弾ませているみたいです。といっても、みんな見当もつかないので、でっかいカラスだとか、そういう天気なんだとかとりあえずいろんな説をお互いに言い合っています。\\

すると、ほどなくして狩人のお兄さんが切迫した表情で村長の家に駆け込んでいきました。\\
気になった私は後を追って村長の家に行きます。\\

\noindent
「村長、大変です!!外をご覧になられましたか!?」\\
「いいや、見とらん。リリアに様子を見に行かせたが……」\\
「竜です」\\
「なんじゃと?」\\
「竜の群れです!!竜の群れが来ます!!!」\\
「なっ……なんじゃと!?お前今なんと言いおった!!!?」\\
「ですから!!目視で百は確認できる竜の群れが、この森めがけてやってきています!!!」\\

その時の私は、その言葉の現実味を感じられず、ただただ二人の会話を聞いているだけでした。しかも、ふたりの会話の内容もほとんど入ってきません。そのうち、長老が私に気付くなり、『何をぼーっとしとる!聞いておっただろ!はよう逃げる準備をせんか!!!』と怒鳴ったことで、ようやく実感と恐怖を感じ始めました。\\

村長の指示通りに家に帰りお母さんに話を伝えると、お母さんはすぐさま貴重品をまとめて私を連れ出しました。

家から出た頃には村長から話を聞いた村人たちで騒ぎになっており、戦える者達は村やみんなを守るために防衛線を張り、残りの村人たちは北の林道へとひたすらに逃げていきます。中には生まれ育った村で最後まで過ごしたいという人もいて、村が好きな気持ちがわかる私たちはそう言う人たちを止めることはできませんでした。\\

私達はひたすら北へ北へと進みます。村の馬の数は限られているので、馬に乗れない人達は基本的にはみんな徒歩で逃げ続けます。でも、竜の速さは人の足や馬の速さを圧倒的に超えていて、空に浮かぶ黒い斑点が明確に竜の輪郭へと変じていくなか、まだ対峙していないにも関わらず、刻一刻と迫る恐怖に足が竦んで動けない人までいました。\\

後ろからは、既にけたたましい咆哮と、大地を揺らす地鳴り、爪と剣がかち合う剣戟の音、そしてこの森を守ろうとする狩人たちの雄叫びと絶叫が聞こえてきます。それらを振り切るかのように、お母さんは私に言いつけました。\\

\noindent
「リリア、決して後ろを振り返っては駄目よ」\\

その意味は明白です。つまりーー私たちの後ろで起こることを全て見過ごせということ。

考えなくてもわかります。無力な私は竜に立ち向かっても傷一つつけることできずに消し炭にされるでしょう。村の狩人たちならもしかしたら何とかできるかもしれませんが、それでも犠牲が出ないなんてそんな都合のいい話が通るほど流石に現実が甘くないのは想像に難しくありません。

私たちでは今から起こることはどうすることもできません。たとえ、それがものすごく悲しくて、ものすごく嫌な事でも、私たちには止めることは出来ないのです。

強いて私にできることを挙げれば、竜たちから逃げ延びることぐらいですが、それも叶うかどうか分からないのです。

この頃の私は既に、そんなことも分からないような子供ではありませんでした。\\

あとはただ、祈るだけ。

少しでも、惨いことにならないよう、祈るだけ。

その始まりが来ないことを、あるいはそれが少しでも早く終わることを、祈るだけ。

祈って、祈って、そして、ひたすら逃げ続けるだけ。\\

それで何になるのかは分からない。誰に祈ってるのかも分からない。けれど、本当に私に出来ることはそれぐらいしかありませんでした。森の守り神でもなく、国の教えの神でもなく、ただ、ただ、"神様"に祈るのです。救ってくれるか分からない"神様"に祈るのです。\\

――だからといって、救ってくれるわけではないのですが。\\

それからそう時間の経たないうちに――奴らは私たちの前に現れました。バサリバサリと私たちの頭上を過ぎ去り、わざわざ私たちの進行方向に立ちふさがった3体の漆黒の竜は、夜の闇を煮詰めたような瞳でギロリとこちらを見つめながら、頭に響くような低音で喉を鳴らします。それはまるで、舌なめずりをしながら今にも食らってやろうというかのように。それはまるで、憎しみに狂ってしまった復讐の鬼が仇を見つけて猛る怒りをぶつけるかのように。

それと目が合ってしまった私は、あまりの恐怖と威圧感に、金縛りに遭ってしまったかのように動けなくなってしまいました。\\

――逃げないと

――逃げないと\\

そう体に言い聞かせても、相手はちっとも言うことを聞いてくれません。\\

――逃げなければ、死ぬ\\

そう思えば思うほど、自分の体は恐怖に竦み、意識は恐怖で染め上げられていきます。

私が自分から逃げることは無いと悟った竜は、遂にその足をこちらへと進め始めました。ゆっくりと、しかし確実に。黒い影は私へと迫っていきます。\\

その時、次に動き出したのは、お母さんでした。完全に竜の不意を突いたお母さんは、手を引いて私を森の中へと引き入れます。\\

それを皮切りに、他のみんなも思い思いの方へ駆け出しました。しかし、竜たちも同じ手は食いません。3体の内1体はこちらへ、他の2体や後から来た竜たちはその場にいた人たちを襲い始めました。

絶叫、衝撃、そして、人の体からは絶対に出てはいけない音。どれも全部、絶対に聞きたくなかった音。そんなものを背景音に、私たちは追いかけてくる竜から、ひたすら逃げて、逃げて、逃げ続けます。\\

幸い、森は多くの木々によって覆いつくされており、それらが竜の進行を妨げます。竜はひたすら森の木々をなぎ倒しながら、私たちを執拗に追い続けるのです。

しかし、森が妨げるのは竜の進行だけではありません。私たちの足は、生い茂る低木たちによってどんどん傷つけられています。でも、そんなことを気にする余裕すら、今の私たちには許されていませんでした。\\

――竜が諦めるのが先か。私たちが力尽きるのが先か。\\

そんな時、私の手を引いていた手が、突然私の手から離れました。お母さんは木の根に躓き、倒けてしまったのです。\\

「お母さん!?」\\

お母さんは体をくの字に浮かせた不自然な倒れ方をしていました。私が手を引き、起き上がらせようとしますが、上手くいきません。力ずくで引っ張ると、お母さんは苦悶の声を上げます。\\

それで私は気付きました。お母さんは低木の枝が倒けた拍子に刺さってしまったのです。\\

\noindent
「振り返るなと言ったでしょう……私は置いて……逃げなさい」\\
「そんなの出来ないよ!!」\\

私はこの時、なんでこんな時に限ってお母さんではなく私が倒けなかったのかと心底思いました。私は必死で、かつ慎重に、少しでもお母さんを傷つけないように低木の枝を引っこ抜こうとします。

森の木々のおかげで離れていた竜との距離もすっかり埋まってしまいました。\\

――やっぱり、"神様"は何もしてくれません。\\

全部を諦めて、手を止めようとしたその時――\\

\noindent
「俺の家族に――手を出すな」\\

一閃。背後からの一太刀に当然怯む竜。振り返らせる暇を与えず、前に回ってさらに一刀。竜はぐらりとその体勢を崩します。紫電のごとき速さで竜に剣を浴びせたその人物は、竜の前に立ちふさがった後、お母さんに刺さっている枝をすかさず切り落としました。\\

\noindent
「大丈夫かい?リリア」\\
「お父さん!!私は平気だけど、お母さんが!!」\\
「分かってる。ユリアーネに刺さった枝が抜けたらこれを飲ませなさい」\\

そう言われて渡されたのは、緑色の液体が入った小さな小瓶でした。お父さんの持っている治療薬なら上等品であるはずなので、きっとお母さんの傷もたちどころに治せるはずです。

それを受け取った私は、お母さんを仰向けに寝かせ、今度こそ刺さった枝を抜き始めます。\\

\noindent
「私は……いいから……グラド……リリアを連れて……」\\
「何バカなこと言ってるんだいユリア。3人で逃げるに決まってるだろう?」\\

お父さんは当たり前のように言い放ちます。でもこれは口だけではなく、きっと勝算あっての言葉だったと感じました。

お父さんは、王国騎士団の隊の副隊長を務める実力者で、戦況を見る目や冷静な判断力に長けた人です。難しいことに挑んだとしても、救えるはずのものをむやみに手放すようなことはしません。仮にもし私とお母さん、どちらも助けることが難しいのなら、お父さんはきっと、私を連れて逃げたと思います。\\

お父さんは、お母さんに向けていた顔を竜の方に向け直し剣を構えると、その頃には竜も体勢を立て直していました。双方向かい合い、呼吸を整え、互いの間合いを取ります。

ひりついた空気に気圧されないよう気を保ちながら、私はお母さんに刺さった枝を傷を広げないように丁寧に抜いていきました。\\

\noindent
「無辜の民を傷つけ、その安寧を奪いし竜よ。騎士の誇りに懸け、今ここに名乗らせてもらう。我が名グラディウス・クレール。村のため、家族のため、そして大義のため――お前を討つ」\\

その名乗りを合図に、お互いに距離を詰めて。


\end{document}