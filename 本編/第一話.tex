\documentclass[oneside, a5paper]{jsbook}
\usepackage{bxpapersize}
\usepackage{pxrubrica}
\usepackage{okumacro}
\usepackage{titlesec}
\usepackage{epigraph}
\renewcommand{\postchaptername}{\if@english\else 話\fi}

\begin{document}


\chapter{はじまり}
\epigraph{あの満天に広がる星空は、きっと女神様の魔法だったんです――}{リリアーナ・クレール}

\section{プロローグ}

その人は\kenten{アレ}の前に立ち塞がりました。\\

当時の私はアレが何なのか全然知らなかったけど、きっととても強大で邪悪なものだということは分かりました。\\
あの時、アレは――悪竜の群れは明らかに悪意を以って、森を焼き、家屋を踏み潰し、人を食らい、命を蹂躙していたのです。\\
悪竜は女の人も、子供も、お爺ちゃんお婆ちゃんも見境なく殺していきました。
村の腕利きの人たちも、悪竜から皆を守って死にました。
仲のいい友達も何人も居なくなって、私のお父さんもお母さんを庇って\dots 。\\

その時です、\kenten{その人}は私たち前に現れ、アレとの間に立ちふさがりました。\\
その人が張った結界によって、村は不思議な守りに包まれます。\\
あらゆる一切の邪悪、不条理、暴虐を断つ、清浄なる夢幻の守り。\\
それは悪竜であっても例外なく防ぎました。

悪竜は困惑します。\\
『この結界は何なのか、何がここまで強力な結界を作り上げたのか。』

そんな中、その人は臆することなく悪竜の群れへと突き進みます。\\
その人の杖の一振りは、青白い弧を描き、その軌跡は一条の星となって次々と悪竜を撃ち落としていきます。

一匹、また一匹と打ち落とされていく中、悪竜は理解します。\\
『自分たちを打ち落としているのは、あの村を守っているのは、その人なのだ。』と。\\

悪竜の群れはあの人を倒さんと向かっていきます。\\
ある悪竜は鋭い牙で食らいつきました。\\
ある悪竜は鉄より硬い爪を突き立てました。\\
ある悪竜は灼熱の業火で焼き払いました。\\
――数百の悪竜がその人へと攻めかかりました。\\

それでも――それでも、その人は怯みません。\\
道を阻む悪竜を次々と打ち砕き――\\
一歩、また一歩と群れの中心へと突き進んでいきました。\\

そして――その悪竜の群れの主と相対するのです。\newpage
――その時は来ました。\\
それは、星空そのものでした。\\

「其は虚無の海に輝く星。途絶えず、揺るがず、違えず、人を照らす希望の光」\\\\
その時の光景を忘れたことはありません。\\
真昼の空に広がるのは満天の星空。\\
今まで見たどんな景色より、それは美しく鮮烈で、そして何より――\\

「邪悪なるものを撃ち滅ぼし、人の世に安寧と救済を――!!」\\\\
私の心を掴んで離しませんでした。\\
だから私は――


\end{document}