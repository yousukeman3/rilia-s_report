\documentclass[oneside, a4paper]{jsbook}
\usepackage[textwidth=50zw,lines=43]{geometry}
\usepackage{bxpapersize}
\usepackage{pxrubrica}
\usepackage{okumacro}
\usepackage{titlesec}
\usepackage{epigraph}
\pagestyle{headings}
\renewcommand{\postchaptername}{\if@english\else 話\fi}

\begin{document}


\chapter{はじまり}
\epigraph{あの満天に広がる星空は、きっと女神様の魔法だったんです――}{リリアーナ・クレール}

\section{プロローグ}

その人は\kenten{アレ}の前に立ち塞がりました。\\

当時の私はアレが何なのか全然知らなかったけど、きっととても強大で邪悪なものだということは分かりました。\\
あの時、アレは――悪竜の群れは明らかに悪意を以って、森を焼き、家屋を踏み潰し、人を食らい、命を蹂躙していたのです。\\
悪竜は女の人も、子供も、お爺ちゃんお婆ちゃんも見境なく殺していきました。
村の腕利きの人たちも、悪竜から皆を守って死にました。
仲のいい友達も何人も居なくなって、私のお父さんもお母さんを庇って\dots 。\\

%\newpage

その時です、\kenten{その人}は私たちの前に現れ、アレとの間に立ちふさがりました。\\
その人が張った結界によって、村は不思議な守りに包まれます。\\
あらゆる一切の邪悪、不条理、暴虐を断つ、清浄なる夢幻の守り。\\
それは悪竜であっても例外なく防ぎました。

悪竜は困惑します。\\
『この結界は何なのか、何がここまで強力な結界を作り上げたのか。』

そんな中、その人は臆することなく悪竜の群れへと突き進みます。\\
その人の杖の一振りは、青白い弧を描き、その軌跡は一条の星となって次々と悪竜を撃ち落としていきます。

一匹、また一匹と打ち落とされていく中、悪竜は理解します。\\
『自分たちを打ち落としているのは、あの村を守っているのは、あの人間なのだ。』と。\\

%\newpage

悪竜の群れはその人を倒さんと向かっていきます。\\
ある悪竜は鋭い牙で食らいつきました。\\
ある悪竜は鉄より硬い爪を突き立てました。\\
ある悪竜は灼熱の業火で焼き払いました。\\
――数百の悪竜がその人へと攻めかかりました。\\

それでも――それでも、その人は怯みません。\\
道を阻む悪竜を次々と打ち砕き――\\
一歩、また一歩と群れの中心へと突き進んでいきました。\\

そして――その悪竜の群れの主と相対するのです。\\%\newpage
――その時は来ました。\\
それは、星空そのものでした。\\

\noindent
「其は虚無の海に輝く星。途絶えず、揺るがず、違えず、人を照らす希望の光」\\

その時の光景を忘れたことはありません。\\
真昼の空に広がるのは満天の星空。\\
今まで見たどんな景色より、それは美しく鮮烈で、そして何より――\\\\
「邪悪なるものを撃ち滅ぼし、人の世に安寧と救済を――!!」\\\\
私の心を掴んで離しませんでした。\\
だから私は――
\newpage
\section{巣立ち}

\noindent
「お財布は持った?」\\
「持ったよ!」\\
「お弁当は持った?」\\
「持った!」\\
「\ruby{身分手帳}{クラスブック}は?」\\
「持ったってば!!」\\
「じゃあ――ティリア先生の推薦書は!?」\\
「そんなの絶対忘れるわけないよ!!」\\

その日は私が村を出る日でした。
家の前には、旅立つ私を見送るために村のみんなが集まっています。\\
家事もだめ、勉強もだめ、狩りも農業も商売もダメな私が初めて頑張れそうな事が出来たって言ったら、村のみんなはすごく喜んでくれました。
村はまだあの事件の傷が癒えていなくて、生活も苦しいはずなのに、学校への入学料は村のみんなで出すって言って、私の遠慮も聞き入れてもらえませんでした。\\
この日見送りに来てくれてた人たちはみんな、こんな身勝手な私の幸せを願ってくれる、私の大好きな人たちです。\\

そんな中、まだかまだかとみんなを待たせて、私とお母さんは何をしていたかと言うと、最後の持ち物チェックをしていました。\\
ドジで間抜けな私は、日ごろからすぐに忘れ物をするので、こうしてお母さんと一緒に持ち物を確認するのは、外出する時の習慣だったのです。\\
でも、それもこれまで。
これからは自分で用意をしなければいけません。(と言っても、結局寮ではエリーと一緒に確認することになるんだけど。)\\

「制服よし、髪型よし、笑顔よし!」\\
私とお母さんで、いつもの確認を一緒に言い終えたら、最後の準備はおしまい。\\
玄関で靴を履いて、いつもの靴ベラとお別れを済ませたら、ドアを開けてみんなの元へ。\\

\noindent
「って、おわっと」\\

……このように躓いてしまうこともあります。\\

\noindent
「いつも歩いてる玄関でよくも器用に躓けるわねぇ」\\
「いったいよぉ……なんで今日に限って躓くのかなぁ」\\
「別に今日だけの話じゃないでしょう。さっさ、皆いるわよ。あいさつしなさい」\\

ちなみにこのように私はいろんな所で躓いてはこけています。(今日に限ってと言うのもただの見栄です。ごめんなさい。)\\

\noindent
「いたそうだなぁリリア。こんな調子じゃァ学園でやってけねぇだろ」\\
「茶化さないであげてよおじさん。リリアーナちゃん、大丈夫?怪我したところあったらすぐに見せてね?」\\

このふたりは鍛冶師のバレルさんと薬師で治療術師のローレルちゃん。ローレルちゃんはよく怪我をする私をお薬や魔術で直してくれます。すごく立派な子で、私とそんなに歳も離れていないのに、既に治癒師の仕事をしっかりとこなしています。道中で怪我をしても大丈夫なように、昨日たんまりとお薬を渡してくれました。

バレルさんもこんな感じではありますがいつも面倒を見てくれている良い人で、学園に入学すると聞くなりスタッフを作ってくれました。スタッフは素材の都合上すごくお金がかかるのに、タダで戴いてしまいました。\\

\noindent
「大丈夫だよローレルちゃん、ちょっと擦っただけだしほとんど怪我になってないから」\\
「おいおい、おじさんにはスルーかよ。ちょっと悲しくなっちまうぜ」\\
「意地悪なことを言うおじさんは知りません」\\
「おじさんも心配しているだけだよ。そう気を悪くしないであげて?」\\
「まぁ、ローレルちゃんがそこまで言うなら……。おじさんもスタッフありがとう、大事にします」\\
「いやぁ、別にそういう風に言わせたかった訳じゃなくってだなァ……いや、気に入ってくれたなら鍛冶屋冥利に尽きるってもんよ。頑張れよリリア!」\\

そういうとおじさんは私の背中を叩きます。(これごまかしたよね?絶対照れ隠し入ってるよね?)\\

\noindent
「これこれ、他にも話したいものは居るんじゃ、貴様らだけでリリアーナを独占するな。」\\
「あ、長老様、ごめんなさい……」\\
「わりぃわりぃ、朝っぱらから心配になるような事しでかすんでついな」\\

この人は長老、村をまとめてくれている方です。(実はみんな長老としか呼ばないから未だに名前を知らないんだよね……。今度調べておきます……。)私が入学したいって言ったら、最初は反対していましたが、最後には納得してみんなに声をかけて入学金を募ったりしてくれました。普段もすごく優しい人で、困ったことがあったら相談に乗ってくれたりしてくれます。\\

\noindent
「全く、朝話し始めると出発が遅くなるから、言いたいことがあるなら昨日のうちに言っておけとあれほど言っておったのにな。ま、いい。リリアーナ、村を代表して儂が祝辞を言わせてもらうぞ」\\
「じいさんは良いのかよ」\\
「儂以外に誰か適任が居るんか?」\\
「いやバルディウスの兄貴とか」\\
「あやつ今居らんやろ。今日も見送りたいと言いながらも仕事で王都じゃろうが。というかそもそもリリアの父親じゃろあやつ。絶対親バカを炸裂させて終わりじゃ」\\
「まとめ役っつったら浮かんだ名前を言っただけなんだけどな……」\\
「うるさいうるさい。お前と話しとったら埒が明かんわ。」\\

此処までの話で分かると思いますが、私の村の人はみんな仲良しです。誰かのピンチはみんなで乗り越えて、誰かの嬉しいことはみんなでお祝いします。私はこの時そんなみんなに沢山助けてもらって、このチャンスにありつけました。(ちなみにお父さんの話は後でするからもうちょっと待ってね)\\

\noindent
「こほん、では改めて。リリアーナ・クレール、この度は村の――」\\
「ごめんね村長、今日は長い奴はいいなかぁ~?なんて」\\
「コラ、お前はまたそうやって……はぁ、ま確かに、今更お前にこういう事を言ってもお前からすれば門出が遅くなるだけなんじゃろ……」\\
「そういう事だよ!流石村長、話が分かるね!」\\
「バカモン!皮肉なのが分らんのか……はぁ。では短く纏めるぞ、村のモンはみんなお前が大きくなって帰ってくるのを応援しとる。やるべきことはしっかりとこなしながら、のびのび勉強するんじゃぞ」\\
「うん!みんなありがとう!!私行ってくる!!」\\

この日は行商人のダリアおじさんが街の方に帰る日なので、その馬車の荷台に乗せてもらいます。\\

\noindent
「ほら、皆とお別を済ませたらさっさと乗りなお嬢ちゃん」\\
「うん!!それじゃあ行ってくるね、みんな」\\
「いってらっしゃい、リリア。馬車に乗るときに体勢崩してこけたりしなさんなよ」\\
「さすがにだいじょ――おっとっと」\\
「ほら言わんこっちゃない」\\

こうして、私は村を出ます。この一歩が、私の人生をどうやって変えるのかは、まだこの時にはよくわからないけど、新しい場所、新しい人とたくさんの発見が待ち構えていることは言うまでもなく、この時の私は期待に胸を膨らませるのでした。



\end{document}